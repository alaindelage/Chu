% \documentclass[a4paper]{report}
\input{preamble}
\input{hyphenation}

%\setlength{\baselineskip}{2ex}
% \renewcommand{\baselinestretch}{1.2}
\selectlanguage{ngerman}
\usepackage{fourier}

\begin{document}
% \doublespacing
% \onehalfspacing
\thispagestyle{empty}

% einzeiliger Titel

{\Large Andrea Long Chu\ \ \emph{Females} (2019)}

% oder mehrzeiliger Titel

% \begin{tabbing}
% 	{\Large AUTOR} \ \ \={\Large \emph{TITEL}}\\[1ex]
% 	\> {\Large (JAHR)}
% \end{tabbing}
\vskip 5ex


% \flushleft
\RaggedRight

% \changefont{ptm}{m}{n}

\begin{center}
\emph{BONGI. I’m so female I’m subversive.}
\end{center}

\begin{quote}
	In fact, while the SCUM Manifesto is often taught in university courses as a feminist text, it’s not at all clear whether this label is appropriate.
\end{quote}
\vskip 2ex

\begin{quote}
	\textbf{For the record, I’m not sure if what you’re reading is a feminist text, either. I’m not sure if I want it to be.}
\end{quote}
\vskip 3ex

\begin{center}
	\emph{BONGI. Eventually the expression “female of the species”’ll be a redundancy.}
\end{center}

femaleness: a universal existential condition, defined by self-negation (“a wildly tendentious definition”)
\begin{quote}
	The thesis of this little book is that \textbf{femaleness is a universal sex defined by self-negation, against which all politics, even feminist politics, rebels}. Put more simply: \textbf{Everyone is female, and everyone hates it}.\\[1ex]

	Some explanations are in order. For our purposes here, I’ll define \textbf{as \emph{female} any psychic operation in which the self is sacrificed to make room for the desires of another}. These desires may be real or imagined, concentrated or diffuse---a boyfriend’s sexual needs, a set of cultural expectations, a literal pregnancy---but in all cases, \textbf{the self is hollowed out, made into an incubator for an alien force}. To be female is to \textbf{let someone else do your desiring for you, at your own expense}. This means that \textbf{femaleness, while it hurts only sometimes, is always bad for you}. Its ultimate toll, at least in every case heretofore recorded, is death.\\[1ex]

	Clearly, this is \textbf{a wildly tendentious definition}. It’s even more far-fetched if you, like me, are applying it to everyone---literally everyone, every single human being in the history of the planet. So it’s true: \textbf{When I talk about females, I am not referring to biological sex, though I’m not referring to gender, either}. I’m referring instead to \textbf{something} that might as well be sex, the way that reactionaries describe it (permanent, unchanging, etc.), but \textbf{whose nature is ontological, not biological}. Femaleness is not an anatomical or genetic characteristic of an organism, but rather \textbf{a universal existential condition}, the \textbf{one and only structure of human consciousness. To be is to be female: the two are identical}.
\end{quote}
\vskip 2ex

gender: how one copes with being female
\begin{quote}
	Everyone is female, but \textbf{how one \emph{copes with} being female}---\textbf{the specific defense mechanisms that one consciously or unconsciously develops as a reaction formation \emph{against} one’s femaleness}, within the terms of what is historically and socioculturally available---this is what we ordinarily call \emph{gender}.
\end{quote}
\vskip 2ex

the dawning realization that one’s desires are not one’s own --- politics: in its essence, anti-female
\begin{quote}
	This brings me to the second part of my thesis: \textbf{Everyone is female---and everyone hates it}.\\[1ex]

	By the second claim, I mean something like what Valerie meant: that human civilization represents a diverse array of \textbf{attempts to suppress and mitigate femaleness}, that this is in fact the implicit purpose of all human activity, and, most of all, \textbf{that activity we call politics}.\\[1ex]

	\textbf{The political is the sworn enemy of the female; politics begins, in every case, from the optimistic belief that another sex is possible}. This is the root of all political consciousness: \textbf{the dawning realization that one’s desires are not one’s own, that one has become a vehicle for someone else’s ego}; in short, t\textbf{hat one is female, but wishes it were not so}. \textbf{Politics is, in its essence, anti-female}.
\end{quote}
\vskip 2ex

To be for women, imagined as full human beings, is always to be against females [‘feminism against females’]
\begin{quote}
	Perhaps the oldest right-wing accusation brought by men and other women against feminists, whether they demanded civic equality or anti-male revolution, was that \textbf{feminists were really asking, quite simply, not to be women anymore}.\\[1ex] 
	There was a kernel of truth here: Feminists didn’t want to be women anymore, at least under the existing terms of society; or to put it more precisely, \textbf{feminists didn’t want to be female anymore}, either advocating for the abolition of gender altogether or proposing new categories of womanhood unencumbered by femaleness. \textbf{To be for women, imagined as full human beings, is always to be against females}. In this sense, feminism opposes misogyny precisely inasmuch as it also expresses it.\\[1ex]
	Or maybe I’m just projecting.
\end{quote}
\vskip 3ex

\begin{center}
	\emph{BONGI. You’re wrong---I’m not a watcher; I’m a woman of action.}
\end{center}

\begin{quote}
	At an academic event, I was once asked what I had meant by the term ethics as I’d used it in a publication. I hesitated and then I said, “I think I mean commitment \textbf{to a bit}.” [\dots\hspace{-0.3ex}]\\[1ex]
	In stand-up comedy, a bit is a comic sequence or conceit, often involving a brief suspension of reality. \textbf{To commit to a bit is to play it straight—that is, to take it seriously}. A bit may be fantastical, but the seriousness required to commit to it is always real. This is the humorlessness that vegetates at the core of all humor. That’s what makes the bit funny: the fact that, for the comic, it isn’t.\\[2ex]

	This would become the first principle of the SCUM Manifesto: Valerie would make statements not because they were accurate or provable, but simply because she \emph{wanted to}. Readers would be \textbf{confronted by desire, not truth}, peeking out of the text like a tattoo from a sleeve---a reminder of the flesh behind every idea.\\[2ex]

	The paradox of the manifesto---and I’m convinced that Valerie knew this---is that its call to action is just that: \textbf{a call, not an act}, desire spilling over the lip of the text like too much liquid. It’s too serious to be taken seriously.\\[2ex]

	I thought at first of writing this book, after Valerie, in the style of a manifesto—short, pointed theses, oracular, and outrageous. We share this, I think: \textbf{a preference for indefensible claims, for following our ambivalence to the end}, for screaming when we should talk and laughing when we should scream.\\[2ex]

	I don’t really want to tell anyone what to do; I want to be \emph{told}.
\end{quote}
\vskip 3ex

\begin{center}
	\emph{GINGER. She has penis envy. She should see an analyst.}
\end{center}

the castration complex: in fact, the fear that one, having been castrated, \emph{will like it}
\begin{quote}
	When I say that everyone is female, I am simply restating something psychoanalytically uncontroversial---namely, that castration happens \emph{on both sides}.\\[2ex]

	Indeed, the castration complex is easily mistaken for the fear that one will be castrated; in fact, it is the fear that one, having been castrated, \emph{will like it}. \textbf{Pussy envy} is therefore not the mutually exclusive opposite of penis envy, but \textbf{a universal desire atop which the latter develops as a reaction formation: Everyone does their best to want power, because deep down, no one wants it at all}.
\end{quote}
\vskip 3ex

\begin{center}
	\emph{BONGI. Hell’o, Gorgeous.}
\end{center}

female: becoming a canvas for someone else’s fantasy --- dumb blonde, in the most technical sense
\begin{quote}
	But \textbf{Gigi Gorgeous repels depth. She rests delicately on the surface of things}, like a water skipper, never sinking.\\[2ex]
	
	At the heart of Gorgeous’s body of work, not to mention her actual body, is a rigorous, compulsive submission to technique: the stroke of a contouring brush, the precise curve of a breast. If it’s not perfect, it must be done again. It’s not just that conventional beauty standards require Gorgeous to use these techniques to be recognized as a woman, though this is certainly true. \textbf{It’s that the very fact of her submission to them is female. Gender transition, no matter the direction, is always a process of becoming a canvas for someone else’s fantasy}. You \textbf{cannot be gorgeous without someone to be gorgeous \emph{for}}. To achieve this, Gorgeous has \textbf{sanded her personality down} to the bare essentials. \textbf{She laughs at what is funny, she cries at what is sad, and she is miraculously free of serious opinions}. She has become, \textbf{in the most technical sense of this phrase, a dumb blonde}.
\end{quote}
\vskip 3ex








\end{document}