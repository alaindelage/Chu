% \documentclass[a4paper]{report}
\input{preamble}
\input{hyphenation}

%\setlength{\baselineskip}{2ex}
% \renewcommand{\baselinestretch}{1.2}
\selectlanguage{ngerman}
\usepackage{fourier}
\usepackage[colorlinks=true,urlcolor=blue]{hyperref}

\begin{document}
% \doublespacing
% \onehalfspacing
\thispagestyle{empty}

% einzeiliger Titel

{\Large Andrea Long Chu\ \ \emph{Females} (2019)}

% oder mehrzeiliger Titel

% \begin{tabbing}
% 	{\Large AUTOR} \ \ \={\Large \emph{TITEL}}\\[1ex]
% 	\> {\Large (JAHR)}
% \end{tabbing}
\vskip 5ex


% \flushleft
\RaggedRight

% \changefont{ptm}{m}{n}

\begin{center}
\emph{BONGI. I’m so female I’m subversive.}
\end{center}

\begin{quote}
	In fact, while the SCUM Manifesto is often taught in university courses as a feminist text, it’s not at all clear whether this label is appropriate.
\end{quote}
\vskip 2ex

\begin{quote}
	\textbf{For the record, I’m not sure if what you’re reading is a feminist text, either. I’m not sure if I want it to be.}
\end{quote}
\vskip 3ex

\begin{center}
	\emph{BONGI. Eventually the expression “female\\ 
	of the species”’ll be a redundancy.}
\end{center}

femaleness: a universal existential condition, defined by self-negation (“a wildly tendentious definition”)
\begin{quote}
	The thesis of this little book is that \textbf{femaleness is a universal sex defined by self-negation, against which all politics, even feminist politics, rebels}. Put more simply: \textbf{Everyone is female, and everyone hates it}.\\[1ex]

	Some explanations are in order. For our purposes here, I’ll define \textbf{as \emph{female} any psychic operation in which the self is sacrificed to make room for the desires of another}. These desires may be real or imagined, concentrated or diffuse---a boyfriend’s sexual needs, a set of cultural expectations, a literal pregnancy---but in all cases, \textbf{the self is hollowed out, made into an incubator for an alien force}. To be female is to \textbf{let someone else do your desiring for you, at your own expense}. This means that \textbf{femaleness, while it hurts only sometimes, is always bad for you}. Its ultimate toll, at least in every case heretofore recorded, is death.\\[1ex]

	Clearly, this is \textbf{a wildly tendentious definition}. It’s even more far-fetched if you, like me, are applying it to everyone---literally everyone, every single human being in the history of the planet. So it’s true: \textbf{When I talk about females, I am not referring to biological sex, though I’m not referring to gender, either}. I’m referring instead to \textbf{something} that might as well be sex, the way that reactionaries describe it (permanent, unchanging, etc.), but \textbf{whose nature is ontological, not biological}. Femaleness is not an anatomical or genetic characteristic of an organism, but rather \textbf{a universal existential condition}, the \textbf{one and only structure of human consciousness. To be is to be female: the two are identical}.
\end{quote}
\vskip 2ex

gender: how one copes with being female
\begin{quote}
	Everyone is female, but \textbf{how one \emph{copes with} being female}---\textbf{the specific defense mechanisms that one consciously or unconsciously develops as a reaction formation \emph{against} one’s femaleness}, within the terms of what is historically and socioculturally available---this is what we ordinarily call \textbf{\emph{gender}}.
\end{quote}
\vskip 2ex

the dawning realization that one’s desires are not one’s own --- politics: in its essence, anti-female
\begin{quote}
	This brings me to the second part of my thesis: \textbf{Everyone is female---and everyone hates it}.\\[1ex]

	By the second claim, I mean something like what Valerie meant: that human civilization represents a diverse array of \textbf{attempts to suppress and mitigate femaleness}, that this is in fact the implicit purpose of all human activity, and, most of all, \textbf{that activity we call politics}.\\[1ex]

	\textbf{The political is the sworn enemy of the female; politics begins, in every case, from the optimistic belief that another sex is possible}. This is the root of all political consciousness: \textbf{the dawning realization that one’s desires are not one’s own, that one has become a vehicle for someone else’s ego}; in short, \textbf{that one is female, but wishes it were not so}. \textbf{Politics is, in its essence, anti-female}.
\end{quote}
\vskip 2ex

To be for women, imagined as full human beings, is always to be against females [‘feminism against females’]
\begin{quote}
	Perhaps the oldest right-wing accusation brought by men and other women against feminists, whether they demanded civic equality or anti-male revolution, was that \textbf{feminists were really asking, quite simply, not to be women anymore}.\\[1ex] 
	There was a kernel of truth here: Feminists didn’t want to be women anymore, at least under the existing terms of society; or to put it more precisely, \textbf{feminists didn’t want to be female anymore}, either advocating for the abolition of gender altogether or proposing new categories of womanhood unencumbered by femaleness. \textbf{To be for women, imagined as full human beings, is always to be against females}. In this sense, feminism opposes misogyny precisely inasmuch as it also expresses it.\\[1ex]
	Or maybe I’m just projecting.
\end{quote}
\vskip 3ex

\begin{center}
	\emph{BONGI. You’re wrong---I’m not a watcher; I’m a woman of action.}
\end{center}

\begin{quote}
	At an academic event, I was once asked what I had meant by the term ethics as I’d used it in a publication. I hesitated and then I said, “I think I mean commitment \textbf{to a bit}.” [\dots\hspace{-0.3ex}]\\[1ex]
	In stand-up comedy, a bit is a comic sequence or conceit, often involving a brief suspension of reality. \textbf{To commit to a bit is to play it straight—that is, to take it seriously}. A bit may be fantastical, but the seriousness required to commit to it is always real. This is the humorlessness that vegetates at the core of all humor. That’s what makes the bit funny: the fact that, for the comic, it isn’t.\\[2ex]

	This would become the first principle of the SCUM Manifesto: Valerie would make statements not because they were accurate or provable, but simply because she \emph{wanted to}. Readers would be \textbf{confronted by desire, not truth}, peeking out of the text like a tattoo from a sleeve---a reminder of the flesh behind every idea.\\[2ex]

	The paradox of the manifesto---and I’m convinced that Valerie knew this---is that its call to action is just that: \textbf{a call, not an act}, desire spilling over the lip of the text like too much liquid. It’s too serious to be taken seriously.\\[2ex]

	I thought at first of writing this book, after Valerie, in the style of a manifesto—short, pointed theses, oracular, and outrageous. We share this, I think: \textbf{a preference for indefensible claims, for following our ambivalence to the end}, for screaming when we should talk and laughing when we should scream.\\[2ex]

	I don’t really want to tell anyone what to do; I want to be \emph{told}.
\end{quote}
\vskip 3ex

\begin{center}
	\emph{GINGER. She has penis envy. She should see an analyst.}
\end{center}

the castration complex: in fact, the fear that one, having been castrated, \emph{will like it}
\begin{quote}
	When I say that everyone is female, I am simply restating something psychoanalytically uncontroversial---namely, that castration happens \emph{on both sides}.\\[2ex]

	Indeed, the castration complex is easily mistaken for the fear that one will be castrated; in fact, it is \textbf{the fear that one, having been castrated, \emph{will like it}}. \textbf{Pussy envy} is therefore not the mutually exclusive opposite of penis envy, but \textbf{a universal desire atop which the latter develops as a reaction formation: Everyone does their best to want power, because deep down, no one wants it at all}.
\end{quote}
\vskip 3ex
% \pagebreak

\begin{center}
	\emph{BONGI. Hell’o, Gorgeous.}
\end{center}

female: becoming a canvas for someone else’s fantasy --- dumb blonde, in the most technical sense
\begin{quote}
	But \textbf{Gigi Gorgeous repels depth. She rests delicately on the surface of things}, like a water skipper, never sinking.\\[2ex]
	
	At the heart of Gorgeous’s body of work, not to mention her actual body, is a rigorous, compulsive submission to technique: the stroke of a contouring brush, the precise curve of a breast. If it’s not perfect, it must be done again. It’s not just that conventional beauty standards require Gorgeous to use these techniques to be recognized as a woman, though this is certainly true. \textbf{It’s that the very fact of her submission to them is female. Gender transition, no matter the direction, is always a process of becoming a canvas for someone else’s fantasy}. You \textbf{cannot be gorgeous without someone to be gorgeous \emph{for}}. To achieve this, Gorgeous has \textbf{sanded her personality down} to the bare essentials. \textbf{She laughs at what is funny, she cries at what is sad, and she is miraculously free of serious opinions}. She has become, \textbf{in the most technical sense of this phrase, a dumb blonde}.
\end{quote}
\vskip 3ex

\begin{center}
	\emph{MISS COLLINS. She is, without a doubt, the most\\ 
	garish, tasteless faggot I’ve ever run across.}
\end{center}

TERFs, trans-exclusionary radical feminists:
\begin{quote}
	The classic text here is Janice Raymond’s 1979 book, \emph{The Transsexual Empire: The Making of the She-Male}, whose author famously claimed that “all transsexuals rape women’s bodies by reducing the real female form to an artifact, appropriating this body for themselves.” In Raymond’s telling, instead of rejecting sex-role stereotypes altogether, as any good feminist would do, \textbf{transsexuals simply substitute “one sex-role stereotype for another}.” This makes \textbf{transsexuality a perverse extension of sexual objectification}, “the ultimate, and we might even say the logical, conclusion of male possession of women in a patriarchal society.” “Literally,” writes Raymond, “men here possess women.”
\end{quote}
\vskip 2ex

cf. V. Solanas, \emph{SCUM Manifesto}, section “CONFORMITY” (p. 10):
\begin{quote}
	“Differentness in other men, as well as himself, threatens him; it means \emph{they’re} fags whom he must at all costs avoid, so he tries to make sure that all other men conform.\\[1ex]

	The male dares to be different to the degree that he accepts his passivity and his desire to be female, his fagginess. The farthest-out male is the drag queen, but he, although different from most men, is exactly like all the other drag queens; like the functionalist, he has an identity – he is female. He tries to define all his troubles away – but still no individuality. Not completely convinced that he’s a woman, highly insecure about being sufficiently female, he \textbf{conforms compulsively to the man-made stereotype, ending up as nothing but a bundle of stilted mannerisms}.\\[1ex]

	To be sure he’s a ‘Man’, the male must see to it that the female be clearly a ‘Woman’, the opposite of a ‘Man’, that is, the female must act like a faggot. And Daddy’s Girl, all of whose female instincts were wrenched out of her when little, easily and obligingly adapts herself to the role.” (10f.)
\end{quote}
\vskip 3ex
\pagebreak

\begin{center}
	\emph{RUSSELL. You’re not too bad-looking, or,\\ 
	at least, you wouldn’t be if you’d put\\ 
	a skirt on and look like a woman.}
\end{center}

(becoming female) gender: the self’s gentle suicide in the name of someone else’s desires
\begin{quote}
	\textbf{Everyone is female, and everyone hates it}. If this is true, then \textbf{gender is very simply the form this self-loathing takes in any given case. All gender is internalized misogyny}. A female is one who has eaten the loathing of another, like an amoeba that got its nucleus by swallowing its neighbor.\\[1ex]

	Or, to put a finer point on it: Gender is not just the misogynistic expectations a female internalizes but also \emph{the process of internalizing itself}, \textbf{the self’s gentle suicide in the name of someone \emph{else}’s desires, someone \emph{else}’s narcissism}.
\end{quote}
\vskip 2ex

From the perspective of gender, then, we are all dumb blondes
\begin{quote}
	\textbf{The claim that gender is socially constructed has rung hollow} for decades not because it isn’t true, but \textbf{because it’s wildly incomplete}. Indeed, it is trivially true that a great number of things are socially constructed, from money to laws to genres of literature. \textbf{What makes gender \emph{gender}---the substance of gender, as it were---is the fact that it expresses, in every case, the desires of another}. Gender has therefore a complementary relation to sexual orientation: \textbf{If sexual orientation is basically the social expression of one’s own sexuality, then gender is basically a social expression of someone \emph{else}’s sexuality}.\\[1ex] 
	\textbf{In the former case, one takes an object; in the latter case, one \emph{is} an object}.\\[1ex] 
	\textbf{From the perspective of gender, then, we are all dumb blondes}.
\end{quote}
\vskip 2ex

You do not get to consent to yourself---a definition of femaleness
\begin{quote}
	\textbf{Gender transition begins, after all, from the understanding that how you identify yourself subjectively}---as precious and important as this identification may be---\textbf{is nevertheless on its own basically worthless}.\\[1ex]

	\textbf{If identity were all there were to gender, transition would be as easy as thinking it---a light bulb, suddenly switched on}. Your gender identity would simply exist, in mute abstraction, and no one, least of all yourself, would care.\\[1ex]

	On the contrary, \textbf{if there is any lesson of gender transition}---from the simplest request regarding pronouns to the most invasive surgeries---it’s that \textbf{gender is something other people have to \emph{give} you}.\\[1ex] 

	\textbf{Gender exists, if it is to exist at all, only in the structural generosity of strangers}. When people today say that a given gender identity is “valid,” this is true, but only tautologically so. At best it is a moral demand for possibility, but it does not, in itself, constitute the realization of this possibility. \textbf{The truth is, you are not the central transit hub for meaning about yourself}, and \textbf{you probably don’t even have a right to be. You do not get to consent to yourself}, even if you might deserve the chance.\\[1ex]
	\textbf{You do not get to consent to yourself---a definition of femaleness}.
\end{quote}
\vskip 3ex

\begin{center}
\emph{GINGER. Everybody knows that men have much more respect for women who’re good at lapping up shit.}
\end{center}

\begin{quote}
	“I’m completely attuned to the gripping dynamism of the male mind,” she tells Bongi breathlessly. “I talk to men on their level; I have virile, potent, sophisticated interests---I adore positions of intercourse, Keynesian economics, and I can look at dirty pictures for hours on end.”\\[1ex]
	---Ginger, \emph{Up Your Ass}’s resident Daddy’s Girl
\end{quote}

Jamie Loftus, eating David Foster Wallace’s \emph{Infinite Jest} (2016ff.)
\begin{quote}
	Her [Jamie Loftus’s] body looks like a doll’s, or a mannequin’s. Her face is completely out of the frame.
\end{quote}
\vskip 3ex
% \pagebreak

\begin{center}
	\emph{RUSSELL. You don’t know what a female\\ 
	is, you desexed monstrosity.}
\end{center}

(female) sex v/ gender
\begin{quote}
	As C. Riley Snorton has recently documented, the distinction between biological females and women as a social category, far from a neutral scientific observation, developed precisely in order for the captive black woman to be recognized as female---making Sims’s research applicable to his women patients in polite white society---without being granted the status of social and legal personhood.\\[1ex]
	\textbf{Sex was produced}, in other words, \textbf{precisely at the juncture where gender was denied}. \textbf{In this sense, a female has always been less than a person}.
\end{quote}
\vskip 2ex

androgen-insensitivity syndrome
\begin{quote}
	\begin{quote}
		“Without the competition of a male hormone, testicular estrogen does an excellent job of shaping a female. Her good looks may be so outstanding that they have enabled her, in some instances, to earn a living as a model.
		Such is the power of the Eve who lurks, forever imprisoned, in even the most full-bearded, bass-voiced, heroically androgenized and macho-minded of males!”\\[1ex]
		---John Money, \emph{The Adam Principle} (1993)
	\end{quote}

	Does he even know what he’s saying?
\end{quote}
\vskip 3ex

\begin{center}
	\emph{MISS COLLINS. I face reality, and our\\ 
	reality is that we’re men.}
\end{center}

\begin{quote}
	the \emph{Manifesto of the Futurist Woman}, a 1912 essay written by French artist and dancer Valentine de Saint-Point\par

	Men and women each, Saint-Point claims, can only be whole by integrating both male and female elements, and what’s missing most right now, in both sexes, is virility. It’s a paradoxical conclusion: the only way for women to fulfill themselves will be to undergo masculinization. “Every woman must possess not only feminine virtues, but also masculine ones,” Saint-Point writes, “without which she is a female.” The word here in French is \emph{femelle}, as said of livestock, not people.\par

	\emph{Einstein on the Beach}, Brooklyn Academy of Music:\\[1ex]
	[\dots\hspace{-0.3ex}] a blackout, and after a moment, a long illuminated white bar appeared horizontal on the stage floor [\dots\hspace{-0.3ex}]\\
	as I sat there in the dark, watching this great white erection, it felt like eternity. Valerie would have hated it.\par

	When I got home that night, I probably watched porn. I did that most nights, guiltily hiding out in the shared bathroom, as if my roommates wouldn’t notice: a sad, pretentious boy, furious about rape, hopelessly addicted to pornography. The two things fueled each other: the more righteous I felt in public, the more I could wallow, privately, in my shame. My anger made me angrier; what got me hot got hotter.\par

	It would still be years before it would occur to me that I might be a woman. If the thought had presented itself then, I would have batted it away like an insect.\\[1ex]
	I hated being a man, but I thought that was just how feminism felt. Being a man was my punishment for being a man. Anything else was greed.
\end{quote}
\vskip 3ex
\pagebreak

\begin{center}
	\emph{BONGI. Why’re girls called chicks?\\ 
	After all, men have the peckers.}
\end{center}

\begin{quote}
At the heart of \textbf{the manosphere} lies the conviction that men---paradigmatically, though not always, white men—have lost status in the past fifty years, ultimately thanks to the rise of feminism. To awaken to this fact is to take the red pill---a phrase borrowed from the 1999 film \emph{The Matrix}
\end{quote}

\emph{The Matrix}’ red pill
\begin{quote}
	Many have pointed out online that back in the nineties, prescription estrogen was, in fact, red: the 0.625 mg Premarin tablet, derived in Matrix---like fashion from the urine of pregnant mares, came in smooth, chocolatey maroon. [\dots\hspace{-0.3ex}]\par

	Taken seriously, it suggests that the manosphere red-piller’s resentment of immigrants, black people, and queers is a sadistic expression of his own gender dysphoria. In this reading, he is an abortive man, a beta trapped in an alpha’s body, consumed with the desire to be female and desperately trying to repress it.\\[1ex] 
	His desire to increase his manhood is not primary, but a second-tier defense mechanism. Those around him assume he is a leader, a provider, a president; but his greatest fear is that they are mistaken. He radicalizes---shoots up a school, builds a wall---in order to avoid transitioning, the way some closeted trans women join the military in order to get the girl beaten out of them.
\end{quote}

\begin{quote}
Another, hinted at in a footnote, sounds a lot like the Matrix---a vast virtual reality network that men would willingly plug themselves into as “vicarious livers.” “It will be electronically possible for [men] to tune into any specific female [they want] to and follow in detail her every movement,” Valerie explains, declaring it a “marvelously kind and humane way” for women to treat their “unfortunate, handicapped fellow beings.” [loc. cit, p. 25]\par

\textbf{Isn’t that the whole point of gender---letting someone else do your living for you?}
\end{quote}
\vskip 3ex

\begin{center}
	\emph{BONGI. Come and get it.}
\end{center}

the subreddit \href{https://www.reddit.com/r/TheRedPill/}{r/TheRedPill}

\begin{quote}
	an unexpected reversal of roles: in order for a woman to be sure a man’s worth submitting to, she must first dominate him\par

	Desperate to prove he isn’t a woman, he temporarily becomes one.\par

	For Valerie [Solanas], the single greatest hoax in the history of human civilization was the simple idea that men are men. The patriarchal system of sexual oppression therefore existed not to express man’s maleness, but \textbf{to conceal his femaleness}.\par

	this is the surprising core of the whole Red Pill theory of seduction: never stop begging for it\par

	\textbf{Men are not men. Men are never men.}\\
	In 2018, when the \emph{Guardian} asked \textbf{\emph{Fight Club}} author Chuck Palahniuk what he thought of the film’s popularity on the far right, he replied that the phenomenon reflected “how few options men have in terms of metaphors” for their experience of gender. Asked what he thought would come of the alt-right, he answered that he thought it was too fringe to last. “It might be comparable to Valerie Solanas’s \emph{Society for Cutting Up Men},” he told the interviewer. “The extreme always goes away.
\end{quote}
\vskip 3ex
\pagebreak

\begin{center}
	\emph{BONGI. I star in movies for stag parties.\\
		But I’ve got professional integrity---I\\ 
		only work for the top directors.}
\end{center}

“Pleasure and Danger” conference, Barnard College 1982.

\begin{quote}
	\textbf{Pornography} is what it feels like when you think you have an object, but really the object has \emph{you}. It is therefore \textbf{a quintessential expression of femaleness}.
\end{quote}
\vskip 2ex

\begin{quote}
	At stake in all this was the question that Amber Hollibaugh raised at Barnard: “Is there ‘feminist’ sex? Should there be?” Or to put it bluntly: can women have sex without getting fucked?\par
	Valerie’s answer is still the best one: No, but who cares? “Sex is the refuge of the mindless,” she gripes in the SCUM Manifesto, which isn’t against sex so much as deeply unimpressed by it. “Sex is not part of a relationship,” Valerie writes. “On the contrary, it is a solitary experience, non-creative, a gross waste of time.”
\end{quote}
\vskip 3ex

\begin{center}
	\emph{ALVIN. I guess it’s just the romantic in me.}
\end{center}

Joseph Gordon-Levitt, DON JON (USA 2013)
\begin{quote}
	\textbf{Like all men, Jon watches porn not to have power, but to give it up}.\par
	In short, \textbf{pornography feminizes him}. This is where the film’s implicit theory of pornography---\textbf{call it anti-porn postfeminism}---both joins and splits with those of its forerunners in the sex wars. \emph{Don Jon} basically agrees with the MacKinnonite doctrine that porn is structured by the eroticization of dominance and submission---but it locates this power dynamic not in the sex acted out between the commanding men and degraded women onscreen, but in the sex unfolding \textbf{between the addictive pornographic image and the essentially female viewer it dominates}. [\dots\hspace{-0.3ex}]\par

	\textbf{and never watches porn again}. In \emph{Don Jon}’s concluding montage, Jon and Esther stare into each other’s eyes while Jon’s voiceover describes their new, “two-way” kind of love. “I do lose myself in her,” he confides, “I can tell she’s losing herself in me, and we’re just fucking lost together.”\\[1ex] 
	The film closes with Jon and Esther making gorgeously sunlit love in Jon’s bed, \textbf{each penetrating the other’s eyes with their own in an accelerating series of radiant shot–reverse shots}.\\[1ex]
	\textbf{Neither of them, we are asked to believe, are female}.
\end{quote}
\vskip 3ex

\begin{center}
	\emph{BONGI. Downright perverse.}
\end{center}

sissy porn
\begin{quote}
	Almost every night, for at least a year before I transitioned, I would wait till my girlfriend had fallen asleep and slip out of bed for the bathroom with my phone. I was going on Tumblr to look at something called \textbf{sissy porn}.
\end{quote}
\vskip 2ex

JOI (jerkoff instruction) videos
\begin{quote}
	The whole thing is unusually meta, even for porn: many JOI actresses will explicitly shame viewers for wasting their time masturbating instead of fucking a real woman like herself. Humiliation is therefore a frequent theme. Orgasms are often ruined or withheld entirely; affectations of disgust or amusement at the thought of the viewer’s tiny penis are common.
\end{quote}
\vskip 2ex

transsexuality’s long history of being considered a paraphilia ---\\

Ray Blanchard, \emph{autogynephilia}
\begin{quote}
	“All gender dysphoric males who are not sexually oriented toward men are instead sexually oriented toward the thought or image of themselves as women”\\[1ex]
	---Ray Blanchard, “The Classification and Labeling of Nonhomosexual Gender Dysphorias” (\emph{Archives of Sexual Behavior}, 1989)
\end{quote}
\vskip 2ex

autogynephilia describes \emph{the basic structure of all human sexuality}
\begin{quote}
	What Blanchard hoped to describe with the term \emph{autogynephilia} was, of course, exactly what the \emph{SCUM Manifesto} had described twenty years earlier as the psychological disease shared by \emph{all men}.\\[1ex] 
	Indeed, if everyone is female---and I’m hoping you’re starting to believe that they are---then \textbf{autogynephilia describes not an obscure paraphilic affliction but rather \emph{the basic structure of all human sexuality}}.\\[1ex] 
	This is not just because everyone has an erotic image of themselves as female---they do---but \textbf{the assimilation of any erotic image is, by nature, female}.\\[1ex] 
	\textbf{To be female is, in every case, to become what someone else wants}.\\[1ex] 
	\textbf{At bottom, everyone is a sissy}.
\end{quote}
\vskip 3ex

\begin{center}
	\emph{ARTHUR. Fuck is in the air; it’s overpowering; it\\
		carries you away with it, sucks you right up.}
\end{center}

sissy porn: turning people female
\begin{quote}
	turning people female is exactly what sissy porn says it does
\end{quote}
\vskip 2ex

sissy porn: a kind of metapornography
\begin{quote}
	Captions further instruct viewers to understand that \textbf{the very act of looking at sissy porn itself constitutes an act of sexual degradation}, with the implication that, whether they like it or not, \textbf{viewers will inevitably be transformed into females themselves}.\\[1ex] 
	This makes sissy porn \textbf{a kind of metapornography}, that is, porn about what happens to you when you watch porn. In other words, \textbf{sissy porn takes the implicitly feminizing effect of all pornography (even the most vanilla) and promotes it to the level of explicit content}---often with spectacular results.
\end{quote}
\vskip 2ex

Getting fucked makes you female because fucked is what a female is --- tops are props
\begin{quote}
	\textbf{Getting fucked makes you female because fucked is what a female is}.\\[1ex] 
	At the same time, \textbf{sissy porn remains wholly uninterested in who’s doing the fucking}. Men appear, when they appear, only in fragments: a hand, an ass, a stray leg.\\[1ex] 
	\textbf{Tops are props}; their \textbf{function is purely structural}.\\[1ex] 
	“To call a man an animal is to flatter him,” Valerie writes in SCUM. “He’s a machine, a walking dildo. It’s often said that men use women. Use them for what? Surely not pleasure.”
\end{quote}
\vskip 2ex

sissy porn $\sim$ fetish objects --- a guarantee that the penis will be lost forever
\begin{quote}
	For Freud, the fetish was a clear substitute for the “absent female phallus.” The little boy, traumatized by the discovery that his mother has no penis and fearing lest the same fate befall his own, looks for reassurance to an object that can replace that penis [\dots\hspace{-0.3ex}]\\[1ex]
	The fetish is thus “a token of triumph over the threat of castration and a protection against it.” Yet even Freud knew that the fetish, in disavowing castration, \textbf{thereby implicitly acknowledged it};\\[1ex]
	sissy porn exploits this weakness, \textbf{transforming the fetish from an assurance that the penis will be kept safe into a guarantee that the penis will be lost forever}
\end{quote}
\vskip 2ex

\emph{bimboification} --- the GIF format: a kind of centrifuge for distilling the femaleness to its barest essentials
\begin{quote}
	In fact, \textbf{to be a sissy is always to lose your mind}. The technical term for this is \emph{\textbf{bimboification}}. Captions often instruct viewers to submit themselves to hypnosis, brainwashing, brain-melting, dumbing down, and other techniques for scooping out intelligence. “Why do I like the concept of being a Bimbo?” asks one user. “It’s because my brain is always full. [\dots\hspace{-0.3ex}]” [\dots\hspace{-0.3ex}]\\[1ex]
	The gestures most often looped in GIF format almost always register the evacuation of will: wilting faces, trembling legs, eyes rolled back into heads. Even the \textbf{GIF format} itself communicates this, \textbf{a kind of centrifuge for distilling the femaleness to its barest essentials---an open mouth, an expectant asshole, blank, blank eyes}.
\end{quote}
\vskip 2ex

Sissy porn did make me trans. --- sissy porn as an allegory for all desire as such
\begin{quote}
	\textbf{Sissy porn did make me trans}. At very least it served as \textbf{a neat allegory for my desire to be female}---and increasingly, I thought, \textbf{for all desire as such}.\\[1ex] 
	Too often, feminists have imagined powerlessness as the suppression of desire by some external force, and they’ve forgotten that more often than not, \textbf{desire \emph{is} this external force}.\\[1ex] 
	\textbf{Most desire is nonconsensual; most desires aren’t desired}.\\[1ex] 
	Wanting to be a woman was something that descended upon me, like a tongue of fire, or an infection---
\end{quote}
\vskip 2ex

\begin{quote}
	The implication is obvious: \textbf{No one in their right mind would want to be female}.\par
	\textbf{Which, remember, is all of us}.
\end{quote}
\vskip 3ex

\begin{center}
	\emph{BONGI. Let the guys ram each other in\\
	the ass and leave the women alone.}
\end{center}

BBC, “Big Black Cock” trope
\begin{quote}
	The classic explanation for this fetish is the revolutionary Frantz Fanon’s theory of negrophobia as a kind of murderous envy: the white man, projecting onto the black man the “infinite virility” he worries he lacks, proceeds to revenge himself against the latter, prototypically in the form of lynching. [\dots\hspace{-0.3ex}]\\[1ex]
	The paradox of white supremacy, of course, is that it’s actually an inferiority complex: the white man, who could have just as easily fantasized that the black man’s penis was smaller than his own---it would be fantasy either way, after all---nevertheless opts to imagine himself as a sexual failure, going limp in the presence of the black man’s unlimited sexual potency.
\end{quote}
\vskip 2ex

the true threat: that the black man will remind the white man that he never wanted to be a man in the first place
\begin{quote}
	In other words, the true threat, which in Sissy Mindy’s post becomes an alluring promise, is not that the black man will prevent the white racist from being a man, but \textbf{that the black man will remind the white man that he never wanted to be a man in the first place}.
\end{quote}
\vskip 2ex

hashtag “\#white genocide”
\begin{quote}
	In this form, sissy porn becomes, as it were, the truest version of itself: a parodic expression of the altright’s most repressed sexual fantasies. The cheeky implication appears to be not only that becoming female is a bit like dying, but also that white sissification might constitute a form of erotic reparations for the devastation wrought by chattel slavery.
\end{quote}
\vskip 2ex

females and politics never mix
\begin{quote}
	No one actually expects one blow job to change anything. In a sense, that’s the point: \textbf{females and politics never mix}. After decades of tedious feminist debates over agency, one thing is clear: \textbf{women may be capable of political action, but females never are}.
\end{quote}
\vskip 2ex

\emph{forced feminization} (“forced fem”)
\begin{quote}
	Ultimately, \textbf{the phrase \emph{forced feminization} is redundant}: the \textbf{female is always the product of force}, and \textbf{force is invariably feminizing}.\\[1ex] 
	This is why environments designed to forcibly masculinize their inhabit-ants---college fraternities and the US military come to mind---inevitably end up expressing their \textbf{central contradiction (anyone \emph{forced} to be a man couldn’t possibly \emph{be} a man)} not just through rape and sexual assault (of both men and women by men), but also with a set of \textbf{hazing rituals in which men are forced to undergo feminizing sex acts}.\\[1ex] 
	It’s also why the same contingent of feminists who seek to unmask trans women as male pretenders may almost always be relied upon to cast sex workers as the feminized victims of human traffickers---there is \textbf{too much female among each group and not enough woman}. If sex workers were really women, they would rescue themselves from the sexual objectification that makes them women; \textbf{if trans women were women, they would have the good sense not to be}.
\end{quote}
\vskip 3ex

\begin{center}
	\emph{GINGER. Let your soul sway gently in the void.}
\end{center}

ALC, “Apocalypse Manifesto: Towards a Radioactive Art.”
\vskip 2ex

Alan Kaprow, “Art Which Can’t Be Art” (1986).
\vskip 2ex

Yoko Ono, \emph{Cut Piece} (Carnegie Hall, 1965): a death drift
\begin{quote}
	The term “death drive” is too strong to describe what’s going on in \emph{Cut Piece}; it’s more of \textbf{a death drift}, limp and aimless. Yoko isn’t doing anything, after all; that’s the whole point. \textbf{She’s being done to}. She hasn’t given her consent so much as given up consenting.
\end{quote}
\vskip 2ex

Andrea Long Chu, \emph{Cunt Piece}
\begin{quote}
	At no point did I mention Valerie. The performance was too brief to describe my attraction to her, my obsession with her work. It was exactly the kind of sexual stunt that Valerie both loved and loved to hate: unreadable, vaguely hostile, but also weirdly passive, right at the nexus of SCUM and Daddy’s Girl, where most women, including Valerie, lived. Perhaps I also felt possessive. It was a private show after all, but its audience of one had died thirty years before, probably from emphysema, kneeling on the floor of her room at the Bristol Hotel in San Francisco. Praying, I suppose, to no one.
\end{quote}
\vskip 3ex
\pagebreak

\begin{center}
	\emph{ARTHUR. I \emph{am} terrible, aren’t I?}
\end{center}

Andy, Andi
\begin{quote}
	Andy is my birth name. It’s alien to me now, like an old photograph, or a leg that’s fallen asleep. A few people still use it, but this is my fault. When I first came out, I told people I would be going by Andi. \textbf{When spoken aloud, the living name was identical with the dead one}.
\end{quote}

\end{document}