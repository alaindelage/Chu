% \documentclass[a4paper]{report}
\input{preamble}
\input{hyphenation}

%\setlength{\baselineskip}{2ex}
% \renewcommand{\baselinestretch}{1.2}
\selectlanguage{ngerman}
\usepackage{fourier}

\begin{document}
% \doublespacing
% \onehalfspacing
\thispagestyle{empty}

% einzeiliger Titel

% {\Large AUTOR\ \ \emph{TITEL} (JAHR)}

% oder mehrzeiliger Titel

\begin{tabbing}
	{\Large Andrea Long Chu} \ \ \={\Large \emph{On Liking Women}}\\[1ex]
	\> {\large (\emph{n+1}, Issue 30, Winter 2018)}
\end{tabbing}
\vskip 5ex


% \flushleft
\RaggedRight

% \changefont{ptm}{m}{n}


\begin{quote}
I was the only boy. 47
\end{quote}\vskip 2ex

\begin{quote}
a pit stop for \textbf{greasy highway-exit food} 47
\end{quote}
\vskip 2ex

\begin{quote}
\textbf{the away-game bus} cruising back over the border between one red state and another 47
\end{quote}
\vskip 2ex

\begin{quote}
\textbf{The truth is, I have never been able to differentiate liking women from wanting to be like them}.\\[1ex] 
For years, the former desire held the latter in its mouth, like a capsule too dangerous to swallow. 47
\end{quote}
\vskip 2ex

\begin{quote}
When I trawl the seafloor of my childhood for sunken tokens of things to come, these bus rides are about the gayest thing I can find. They probably weren’t even all that gay. It is common, after all, for high school athletes to try to squash the inherent homoeroticism of same-sex sport under the heavy cleat of denial. But I’m too desperate to salvage a single genuine lesbian memory from the wreckage of the scared, straight boy whose life I will never not have lived to be choosy. 47f.
\end{quote}
\vskip 2ex

\begin{quote}
You didn’t know what was a thing you could study?” “Feminism!” she said, beaming.
48
\end{quote}
\vskip 2ex

\begin{quote}
I took a Women’s Studies course that had only one other man in it. I read desperately, from Shulamith Firestone to Jezebel 49
\end{quote}
\vskip 2ex

Valerie Solanas, \emph{SCUM Manifesto} (1967)\\[1ex]

Solanas: politics begins with an aesthetic judgment
\begin{quote}
Life under male supremacy isn’t oppressive, exploitative, or unjust: it’s just fucking boring. \textbf{For Solanas, an aspiring playwright, politics begins with an aesthetic judgment}. This is because \textbf{male and female are essentially styles for her}, rival aesthetic schools distinguishable by their respective adjectival palettes.\\[1ex] 
Men are timid, guilty, dependent, mindless, passive, animalistic, insecure, cowardly, envious, vain, frivolous, and weak.\\[1ex] 
Women are strong, dynamic, decisive, assertive, cerebral, independent, self-confident, nasty, violent, selfish, freewheeling, thrill-seeking, and arrogant. Above all, women are cool and groovy.
49
\end{quote}
\vskip 2ex

transition recast in aesthetic terms --- not to “confirm” some kind of innate gender identity, but because being a man is stupid and boring
\begin{quote}
\begin{quote}
“\textbf{If men were wise, they would seek to become really female}, would do intensive biological research that would lead to men, by means of operations on the brain and nervous system, being able to be transformed in psyche, as well as body, into women.”
\end{quote}

This line took my breath away. This was a vision of \textbf{transsexuality as separatism}, an image of how male-to-female gender transition might \textbf{express not just disidentification with maleness but disaffiliation with men}.\\[1ex] 
Here, \textbf{transition}, like revolution, was \textbf{recast in aesthetic terms, as if transsexual women decided to transition, not to “confirm” some kind of innate gender identity, but because being a man is stupid and boring}.
50
\end{quote}
\vskip 2ex

\begin{quote}
	Overread, perhaps. 50
\end{quote}
\vskip 2ex

Janice Raymond, \emph{The Transsexual Empire: The Making of the She-Male} (1979)
\begin{quote}
	Janice Raymond, whose 1979 book \emph{The Transsexual Empire: The Making of the She-Male} is a classic of anti-trans feminism. 50
\end{quote}
\vskip 2ex

Solanas’s transphobia
\begin{quote}
	\textbf{Mira Bellwether}, creator of \emph{Fucking Trans Women}, the punk-rock zine that taught the world to muff, wrote a lengthy blog post explaining her misgivings about the event, characterizing the SCUM Manifesto as “potentially the worst and most vitriolic example of lesbian-feminist hate speech” in history. She goes on to charge Solanas with biological essentialism of the first degree, citing the latter’s apparent appeal to genetic science: 
	\begin{quote}
	“The male is a biological accident: the Y (male) gene is an incomplete X (female) gene, that is, it has an incomplete set of chromosomes. In other words, the male is an incomplete female, a walking abortion, aborted at the gene stage.” 
	\end{quote}
	For Bellwether, this is unequivocal proof that everything SCUM says about men, it also says about trans women. 50f.
\end{quote}
\vskip 2ex

“the ‘emotion’ of great historiographic form” (F. Jameson)
\begin{quote}
	Any good feminist bears stitched into the burning bra she calls her heart that tapestry of qualifiers we use to tell one another stories about ourselves and our history: radical, liberal, neoliberal, socialist, Marxist, separatist, cultural, corporate, lesbian, queer, trans, eco, intersectional, anti-porn, anti-work, pro-sex, first-, second-, third-, sometimes fourthwave.\\[1ex] 
	These stories have perhaps less to do with What Really Happened than they do with what Fredric Jameson once called \textbf{“the ‘emotion’ of great historiographic form”} — that is, \textbf{the satisfaction of synthesizing the messy empirical data of the past into an elegant historical arc in which everything that happened could not have happened otherwise}. 52
\end{quote}
\vskip 2ex

We read things, watch things, from political history to pop culture, as feminists and as people ... 
\begin{quote}
	We read things, watch things, from political history to pop culture, as feminists and as people, because we want to belong to a community or public, or because we are stressed out at work, or because we are looking for a friend or a lover, or perhaps because we are struggling to figure out how to feel political in an age and culture defined by a general shipwrecking of the beautiful old stories of history. 53
\end{quote}
\vskip 2ex

TERF, trans-exclusionary radical feminist
\begin{quote}
	In this [sc. Bellwether’s] version of the story, feminism excluded trans women in the past, is learning to include trans women now, and will center trans women in the future. This story’s plausibility is no doubt due to a dicey bit of revisionism implied by the moniker \emph{trans-exclusionary radical feminist}, often shortened to TERF. Like most kinds of feminist, TERFs are not a party or a unified front. Their beliefs, while varied, mostly boil down to a rejection of the idea that transgender women are, in fact, women. [\dots\hspace{-0.3ex}] \\[1ex]
	The actual problem with an epithet like TERF is its historiographic sleight of hand: namely, the erroneous implication that all TERFs are holdouts who missed the third wave, old-school radical feminists who never learned any better. 53
\end{quote}
\vskip 2ex

\begin{quote}
	It’s worth considering whether TERFs, like certain strains of the altright, might be defined less by their political ideology (however noxious) and more by a complex, frankly fascinating relationship to trolling, on which it will be for future anthropologists, having solved the problem of digital ethnography, to elaborate. 54
\end{quote}
\vskip 2ex

Germaine Greer, \emph{The Female Eunuch} (1974) (second-wave, transphobic feminism).
\vskip 2ex

Robin Morgan, editor of the widely influential 1970 anthology \emph{Sisterhood Is Powerful}: drag like blackfacing
\begin{quote}
	“We know what’s at work when whites wear blackface; the same thing is at work when men wear drag.” (\emph{Lesbian Tide}, 1973; apropos transsexual folk singer Beth Elliott at West Coast Lesbian Conference of 1973) 55
\end{quote}
\vskip 2ex

second wave feminism, expanding the scope of feminist critique to everyday life, notably sexuality --- \emph{lesbianism}
\begin{quote}
	In expanding the scope of feminist critique to the terrain of everyday life---a move which produced a characteristically muscular brand of theory that rivaled any Marxist’s notes on capitalism---the second wave had inadvertently painted itself into a corner. If, as radical feminist theories claimed, patriarchy had infested not just legal, cultural, and economic spheres but the psychic lives of \emph{women themselves}, then feminist revolution could only be achieved by combing constantly through the fibrils of one’s consciousness for every last trace of male supremacy---a kind of political nitpicking, as it were. And nowhere was this more urgent, or more difficult, than the bedroom.\\[1ex] 
	Fighting tirelessly for the notion that sex was fair game for political critique, radical feminists were now faced with the prospect of putting their mouths where their money had been. Hence \textbf{Atkinson’s} famous slogan: \textbf{“Feminism is the theory, lesbianism is the practice.”} This was the political climate in which both Elliott and Morgan, as a transsexual woman and a suspected heterosexual woman, respectively, could find their statuses as legitimate subjects of feminist politics threatened by the incipient enshrining, among some radical feminists, of something called \textbf{lesbianism} as the preferred aesthetic form for mediating between individual subjects and the history they were supposed to be making--—call these \textbf{the personal and the political}. 56f.
\end{quote}
\vskip 2ex

lesbianism $\rightarrow$ trans-exclusionary radical feminism
\begin{quote}
	there is a historical line to be traced from \textbf{political lesbianism}, as a specific, by no means dominant tendency within radical feminism, to the contemporary phenomenon we’ve taken to calling \textbf{trans-exclusionary radical feminism} 57
\end{quote}
\vskip 2ex

Sheila Jeffreys, \emph{Gender Hurts. A Feminist Analysis of the Politics of Transgenderism} (Routledge 2014): transgenderism as a ruthless appropriation of women’s experience and existence
\begin{quote}
	Like many TERFs, she believes \textbf{that trans women’s cheap imitations of femininity (as she imagines them) reproduce the same harmful stereotypes through which women are subordinated in the first place}. “Transgenderism on the part of men,” Jeffreys writes in her 2014 book \emph{Gender Hurts}, “can be seen as \textbf{a ruthless appropriation of women’s experience and existence}.”\\[1ex] 
	She is also fond of citing sexological literature that classifies transgenderism as a paraphilia. It is a favorite claim among TERFs like Jeffreys that transgender women are gropey interlopers, sick voyeurs conspiring to infiltrate women-only spaces and conduct the greatest panty raid in military history. 57
\end{quote}
\vskip 2ex

political lesbianism’s, TERF’s dread of desire’s ungovernability
\begin{quote}
	Indeed, at least among lesbians, trans-exclusionary radical feminism might best be understood as gay panic, girl-on-girl edition. [\dots\hspace{-0.3ex}]\\[1ex]
	\textbf{trans-exclusionary feminism has inherited political lesbianism’s dread of desire’s ungovernability}. The traditional subject of gay panic, be he a US senator or just a member of the House, is a subject menaced by his own politically compromising desires: to preserve himself, he projects these desires onto another, whom he may now legislate or gay-bash out of existence. The political lesbian, too, is a subject stuck between the rock of politics and desire’s hard place. 57f.
\end{quote}
\vskip 2ex

lesbianism: not an innate identity, but an act of political will --- being a lesbian was about what got you woke, not wet
\begin{quote}
	As Jeffreys put it in 2015, speaking to the Lesbian History Group in London, political lesbianism was intended as a solution to the all-too-real cognitive dissonance produced by heterosexual feminism: “Why go to all these meetings where you’re creating all this wonderful theory and politics, and then you go home to, in my case, Dave, and you’re sitting there, you know, in front of the telly, and thinking, ‘It’s weird. This feels weird.’”\\[1ex] 
	But \textbf{true separatism doesn’t stop at leaving your husband}. It proceeds, with paranoid rigor, to purge the apartments of the mind of anything remotely connected to patriarchy. \textbf{Desire is no exception}.\\[1ex] 
	\textbf{Political lesbianism is founded on the belief that even desire becomes pliable at high enough temperatures}. For Jeffreys and her comrades, \textbf{lesbianism was not an innate identity, but an act of political will}. This was a world in which \textbf{biology was not destiny}, a world where \textbf{being a lesbian was about what got you woke, not wet}. 58
\end{quote}
\vskip 2ex

We are separatists from our own bodies
\begin{quote}
	It seems never to have occurred to Jeffreys that some of us “transgenders,” as she likes to call us, might opt to transition precisely in order to escape from the penitentiary she takes heterosexuality to be. \textbf{It is a supreme irony of feminist history that there is no woman more woman-identified than a gay trans girl like me}, and that Beth Elliott and her sisters were the OG political lesbians: \textbf{women who had walked away from both the men in their lives and the men whose lives they’d been living}. \textbf{We are separatists from our own bodies.} We are militants of so fine a caliber that we regularly take steps to poison the world’s supply of male biology. 58
\end{quote}
\vskip 2ex

\begin{quote}
	But let’s keep things in perspective. Because of Jeffreys, a few women in the Seventies got haircuts. Because of us, there are literally fewer men on the planet. 58
\end{quote}
\vskip 2ex

trans lesbians as feminist vanguard?
\begin{quote}
	\textbf{That trans lesbians should be pedestaled as some kind of feminist vanguard is a notion as untenable as it is attractive.} In defending it, I would be neglecting what I take to be the true lesson of political lesbianism as a failed project: that nothing good comes of forcing desire to conform to political principle. 59
\end{quote}
\vskip 2ex

a feminist anemone
\begin{quote}
	Desire is, by nature, childlike and chary of government.\\[1ex] 
	The day we begin to qualify it by the righteousness of its political content is the day we begin to prescribe some desires and prohibit others. That way lies moralism only. \textbf{Just try to imagine life as a feminist anemone, the tendrils of your desire withdrawing in an instant from patriarchy’s every touch. There would be nothing to watch on TV.} 59
\end{quote}
\vskip 2ex
\pagebreak

the notion that: transition expresses not the truth of an identity but the force of a desire ---\\
transness as a matter not of who one \emph{is}, but of what one \emph{wants}
\begin{quote}
	It must be underscored how unpopular it is on the left today to countenance \textbf{the notion that transition expresses not the truth of an identity but the force of a desire}.\\[1ex] 
	This would require understanding \textbf{transness as a matter not of who one \emph{is}, but of what one \emph{wants}}.\\[1ex] 
	The \textbf{primary function of gender identity as a political concept}---and, increasingly, a legal one---is \textbf{to bracket}, if not to totally \textbf{deny}, \textbf{the role of desire in the thing we call gender}.\\[1ex] 
	Historically, this results from a wish among transgender advocates to quell fears that trans people, and trans women in particular, go through transition in order to get stuff: money, sex, legal privileges, little girls in public restrooms. 59
\end{quote}
\vskip 2ex

why transition? transition to “be” women?
\begin{quote}
	I \textbf{doubt that any of us transition simply because we want to “be” women}, in some abstract, academic way. I certainly didn’t.\\[1ex] 
	\textbf{I transitioned for gossip and compliments, lipstick and mascara, for crying at the movies,} for being someone’s girlfriend, for letting her pay the check or carry my bags, for the benevolent chauvinism of bank tellers and cable guys, for the telephonic intimacy of long-distance female friendship, for fixing my makeup in the bathroom flanked like Christ by a sinner on each side, for sex toys, for feeling hot, for getting hit on by butches, for that secret knowledge of which dykes to watch out for, for Daisy Dukes, bikini tops, and all the dresses, and, my god, \emph{for the breasts}.
\end{quote}
\vskip 2ex

the problem with desire: we rarely want the things we should
\begin{quote}
	But now you begin to see \textbf{the problem with desire: we rarely want the things we should}.\\[1ex] 
	Any TERF will tell you that most of these items are just the traditional trappings of patriarchal femininity. \textbf{She won’t be wrong, either}.\\[1ex] 
	Let’s be clear: \textbf{TERFs are gender abolitionists}, even if that abolitionism is a shell corporation for garden-variety moral disgust. When it comes to the question of feminist revolution, TERFs leave trans girls like me in the dust, primping. In this respect, someone like Ti-Grace Atkinson, a self-described radical feminist committed to the revolutionary dismantling of gender as a system of oppression, is not the dinosaur; I, who get my eyebrows threaded every two weeks, am.\par
	\textbf{Perhaps my consciousness needs raising. I muster a shrug.} 60
\end{quote}
\vskip 2ex

gender confirmation surgeries are aesthetic practices
\begin{quote}
	These days, \textbf{the belief that getting a vagina will make you into a real woman is retrograde in the extreme}.\\[1ex] 
	Many good feminists still only manage to understand bottom surgery by qualifying it as \textbf{a personal aesthetic choice}: \emph{If that’s what makes you feel more comfortable in your body, that’s great}. This is \textbf{as wrongheaded as it is condescending}.\\[1ex]
	To be sure, \textbf{gender confirmation surgeries are aesthetic practices, continuous with rather than distinct from the so-called cosmetic surgeries}. (No one goes into the operating room asking for an ugly cooch.)\\[1ex] 
	So \textbf{it’s not that these aren’t aesthetic decisions; it’s that they’re not \emph{personal}}. That’s \textbf{the basic paradox of aesthetic judgments: they are, simultaneously, subjective and universal}.\\[1ex] 
	Transsexual women don’t want bottom surgery because their personal opinion is that a vagina would look or feel better than a penis. \textbf{Transsexual women want bottom surgery because \emph{most women have vaginas}}. Call that transphobic if you like---that’s not going to keep me from Chili’s-Awesome-Blossoming my dick. 60f.
\end{quote}
\vskip 2ex

the darker fact: trans women wish there were women, period --- not identity but desire
\begin{quote}
	I am being tendentious, dear reader, because I am trying to tell you something that few of us dare to talk about, especially in public, especially when we are trying to feel political: \textbf{not the fact, boringly obvious to those of us living it, that many trans women wish they were cis women},\\[1ex] 
	but \textbf{the darker, more difficult fact that many trans women \emph{wish they were women, period}}. This is \textbf{most emphatically not something trans women are supposed to want}.\\[1ex] 
	The grammar of contemporary trans activism does not brook the subjunctive. Trans women are women, we are chided with silky condescension, as if we have all confused ourselves with Chimamanda Ngozi Adichie, as if we were all simply trapped in the wrong politics, \textbf{as if the cure for dysphoria were wokeness}.\\[1ex] 
	How can you want to be something you already are? Desire implies deficiency; want implies want. \textbf{To admit that what makes women like me transsexual is not identity but desire} is to admit just \textbf{how much of transition takes place in the waiting rooms of wanting things}, to admit that your breasts may never come in, your voice may never pass, your parents may never call back. 61
\end{quote}
\vskip 2ex

the zero-order disappointment: You want it because you want it
\begin{quote}
	Call this \textbf{the romance of disappointment}. You want something. You have found an object that will give you what you want. This object is a person, or a politics, or an art form, or a blouse that fits. You attach yourself to this object, follow it around, carry it with you, watch it on TV.\\[1ex] 
	One day, you tell yourself, it will give you what you want. Then, one day, it doesn’t. Now it dawns on you that your object will probably never give you what you want. But this is not what’s disappointing, not really. \textbf{What’s disappointing is what happens next: nothing}. \textbf{You keep your object}. You continue to follow it around, stash it in a drawer, water it, tweet at it. It still doesn’t give you what you want---but you knew that.\\[1ex] 
	You have had another realization: \textbf{not getting what you want has very little to do with wanting it}. Knowing better usually doesn’t make it better. \textbf{You don’t want something because wanting it will lead to getting it}. \textbf{You want it because you want it}. This is \textbf{the zero-order disappointment} that \textbf{structures all desire and makes it possible}. After all, if you could only want things you were guaranteed to get, you would never be able to want anything at all. 61
\end{quote}
\vskip 2ex
















\end{document}