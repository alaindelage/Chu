% \documentclass[a4paper]{report}
\input{preamble}
\input{hyphenation}

%\setlength{\baselineskip}{2ex}
% \renewcommand{\baselinestretch}{1.2}
\selectlanguage{ngerman}
\usepackage{fourier}

\begin{document}
% \doublespacing
% \onehalfspacing
\thispagestyle{empty}

% einzeiliger Titel

% {\Large AUTOR\ \ \emph{TITEL} (JAHR)}

% oder mehrzeiliger Titel

\begin{tabbing}
	{\Large Andrea Long Chu} \ \ \={\Large \emph{On Liking Women}}\\[1ex]
	\> {\large (\emph{n+1}, Issue 30, Winter 2018)}
\end{tabbing}
\vskip 5ex


% \flushleft
\RaggedRight

% \changefont{ptm}{m}{n}


\begin{quote}
I was the only boy. 47
\end{quote}\vskip 2ex

\begin{quote}
a pit stop for \textbf{greasy highway-exit food} 47
\end{quote}
\vskip 2ex

\begin{quote}
\textbf{the away-game bus} cruising back over the border between one red state and another 47
\end{quote}
\vskip 2ex

\begin{quote}
\textbf{The truth is, I have never been able to differentiate liking women from wanting to be like them}.\\[1ex] 
For years, the former desire held the latter in its mouth, like a capsule too dangerous to swallow. 47
\end{quote}
\vskip 2ex

\begin{quote}
When I trawl the seafloor of my childhood for sunken tokens of things to come, these bus rides are about the gayest thing I can find. They probably weren’t even all that gay. It is common, after all, for high school athletes to try to squash the inherent homoeroticism of same-sex sport under the heavy cleat of denial. But I’m too desperate to salvage a single genuine lesbian memory from the wreckage of the scared, straight boy whose life I will never not have lived to be choosy. 47f.
\end{quote}
\vskip 2ex

\begin{quote}
You didn’t know what was a thing you could study?” “Feminism!” she said, beaming.
48
\end{quote}
\vskip 2ex

\begin{quote}
I took a Women’s Studies course that had only one other man in it. I read desperately, from Shulamith Firestone to Jezebel 49
\end{quote}
\vskip 2ex

Valerie Solanas, \emph{SCUM Manifesto} (1967)\\[1ex]

Solanas: politics begins with an aesthetic judgment
\begin{quote}
Life under male supremacy isn’t oppressive, exploitative, or unjust: it’s just fucking boring. \textbf{For Solanas, an aspiring playwright, politics begins with an aesthetic judgment}. This is because \textbf{male and female are essentially styles for her}, rival aesthetic schools distinguishable by their respective adjectival palettes.\\[1ex] 
Men are timid, guilty, dependent, mindless, passive, animalistic, insecure, cowardly, envious, vain, frivolous, and weak.\\[1ex] 
Women are strong, dynamic, decisive, assertive, cerebral, independent, self-confident, nasty, violent, selfish, freewheeling, thrill-seeking, and arrogant. Above all, women are cool and groovy.
49
\end{quote}
\vskip 2ex

transition recast in aesthetic terms -- not to “confirm” some kind of innate gender identity, but because being a man is stupid and boring
\begin{quote}
\begin{quote}
“\textbf{If men were wise, they would seek to become really female}, would do intensive biological research that would lead to men, by means of operations on the brain and nervous system, being able to be transformed in psyche, as well as body, into women.”
\end{quote}

This line took my breath away. This was a vision of \textbf{transsexuality as separatism}, an image of how male-to-female gender transition might \textbf{express not just disidentification with maleness but disaffiliation with men}.\\[1ex] 
Here, \textbf{transition}, like revolution, was \textbf{recast in aesthetic terms, as if transsexual women decided to transition, not to “confirm” some kind of innate gender identity, but because being a man is stupid and boring}.
50
\end{quote}
\vskip 2ex

\begin{quote}
	Overread, perhaps. 50
\end{quote}
\vskip 2ex










\end{document}