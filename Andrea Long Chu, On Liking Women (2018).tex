% \documentclass[a4paper]{report}
\input{preamble}
\input{hyphenation}

%\setlength{\baselineskip}{2ex}
% \renewcommand{\baselinestretch}{1.2}
\selectlanguage{ngerman}
\usepackage{fourier}

\begin{document}
% \doublespacing
% \onehalfspacing
\thispagestyle{empty}

% einzeiliger Titel

% {\Large AUTOR\ \ \emph{TITEL} (JAHR)}

% oder mehrzeiliger Titel

\begin{tabbing}
	{\Large Andrea Long Chu} \ \ \={\Large \emph{On Liking Women}}\\[1ex]
	\> {\large (\emph{n+1}, Issue 30, Winter 2018)}
\end{tabbing}
\vskip 5ex


% \flushleft
\RaggedRight

% \changefont{ptm}{m}{n}


\begin{quote}
I was the only boy. 47
\end{quote}\vskip 2ex

\begin{quote}
a pit stop for \textbf{greasy highway-exit food} 47
\end{quote}
\vskip 2ex

\begin{quote}
\textbf{the away-game bus} cruising back over the border between one red state and another 47
\end{quote}
\vskip 2ex

\begin{quote}
\textbf{The truth is, I have never been able to differentiate liking women from wanting to be like them}.\\[1ex] 
For years, the former desire held the latter in its mouth, like a capsule too dangerous to swallow. 47
\end{quote}
\vskip 2ex

\begin{quote}
When I trawl the seafloor of my childhood for sunken tokens of things to come, these bus rides are about the gayest thing I can find. They probably weren’t even all that gay. It is common, after all, for high school athletes to try to squash the inherent homoeroticism of same-sex sport under the heavy cleat of denial. But I’m too desperate to salvage a single genuine lesbian memory from the wreckage of the scared, straight boy whose life I will never not have lived to be choosy. 47f.
\end{quote}
\vskip 2ex

\begin{quote}
You didn’t know what was a thing you could study?” “Feminism!” she said, beaming.
48
\end{quote}
\vskip 2ex

\begin{quote}
I took a Women’s Studies course that had only one other man in it. I read desperately, from Shulamith Firestone to Jezebel 49
\end{quote}
\vskip 2ex

Valerie Solanas, \emph{SCUM Manifesto} (1967)\\[1ex]

Solanas: politics begins with an aesthetic judgment
\begin{quote}
Life under male supremacy isn’t oppressive, exploitative, or unjust: it’s just fucking boring. \textbf{For Solanas, an aspiring playwright, politics begins with an aesthetic judgment}. This is because \textbf{male and female are essentially styles for her}, rival aesthetic schools distinguishable by their respective adjectival palettes.\\[1ex] 
Men are timid, guilty, dependent, mindless, passive, animalistic, insecure, cowardly, envious, vain, frivolous, and weak.\\[1ex] 
Women are strong, dynamic, decisive, assertive, cerebral, independent, self-confident, nasty, violent, selfish, freewheeling, thrill-seeking, and arrogant. Above all, women are cool and groovy.
49
\end{quote}
\vskip 2ex

transition recast in aesthetic terms -- not to “confirm” some kind of innate gender identity, but because being a man is stupid and boring
\begin{quote}
\begin{quote}
“\textbf{If men were wise, they would seek to become really female}, would do intensive biological research that would lead to men, by means of operations on the brain and nervous system, being able to be transformed in psyche, as well as body, into women.”
\end{quote}

This line took my breath away. This was a vision of \textbf{transsexuality as separatism}, an image of how male-to-female gender transition might \textbf{express not just disidentification with maleness but disaffiliation with men}.\\[1ex] 
Here, \textbf{transition}, like revolution, was \textbf{recast in aesthetic terms, as if transsexual women decided to transition, not to “confirm” some kind of innate gender identity, but because being a man is stupid and boring}.
50
\end{quote}
\vskip 2ex

\begin{quote}
	Overread, perhaps. 50
\end{quote}
\vskip 2ex

Janice Raymond, \emph{The Transsexual Empire: The Making of the She-Male} (1979)
\begin{quote}
	Janice Raymond, whose 1979 book \emph{The Transsexual Empire: The Making of the She-Male} is a classic of anti-trans feminism. 50
\end{quote}
\vskip 2ex

Solanas’s transphobia
\begin{quote}
	\textbf{Mira Bellwether}, creator of \emph{Fucking Trans Women}, the punk-rock zine that taught the world to muff, wrote a lengthy blog post explaining her misgivings about the event, characterizing the SCUM Manifesto as “potentially the worst and most vitriolic example of lesbian-feminist hate speech” in history. She goes on to charge Solanas with biological essentialism of the first degree, citing the latter’s apparent appeal to genetic science: 
	\begin{quote}
	“The male is a biological accident: the Y (male) gene is an incomplete X (female) gene, that is, it has an incomplete set of chromosomes. In other words, the male is an incomplete female, a walking abortion, aborted at the gene stage.” 
	\end{quote}
	For Bellwether, this is unequivocal proof that everything SCUM says about men, it also says about trans women. 50f.
\end{quote}
\vskip 2ex

“the ‘emotion’ of great historiographic form”
\begin{quote}
	Any good feminist bears stitched into the burning bra she calls her heart that tapestry of qualifiers we use to tell one another stories about ourselves and our history: radical, liberal, neoliberal, socialist, Marxist, separatist, cultural, corporate, lesbian, queer, trans, eco, intersectional, anti-porn, anti-work, pro-sex, first-, second-, third-, sometimes fourthwave.\\[1ex] 
	These stories have perhaps less to do with What Really Happened than they do with what Fredric Jameson once called \textbf{“the ‘emotion’ of great historiographic form”} — that is, \textbf{the satisfaction of synthesizing the messy empirical data of the past into an elegant historical arc in which everything that happened could not have happened otherwise}. 52
\end{quote}
\vskip 2ex

We read things, watch things, from political history to pop culture, as feminists and as people ... 
\begin{quote}
	We read things, watch things, from political history to pop culture, as feminists and as people, because we want to belong to a community or public, or because we are stressed out at work, or because we are looking for a friend or a lover, or perhaps because we are struggling to figure out how to feel political in an age and culture defined by a general shipwrecking of the beautiful old stories of history. 53
\end{quote}
\vskip 2ex

TERF, trans-exclusionary radical feminist
\begin{quote}
	In this [sc. Bellwether’s] version of the story, feminism excluded trans women in the past, is learning to include trans women now, and will center trans women in the future. This story’s plausibility is no doubt due to a dicey bit of revisionism implied by the moniker \emph{trans-exclusionary radical feminist}, often shortened to TERF. Like most kinds of feminist, TERFs are not a party or a unified front. Their beliefs, while varied, mostly boil down to a rejection of the idea that transgender women are, in fact, women. [\dots\hspace{-0.3ex}] \\[1ex]
	The actual problem with an epithet like TERF is its historiographic sleight of hand: namely, the erroneous implication that all TERFs are holdouts who missed the third wave, old-school radical feminists who never learned any better. 53
\end{quote}
\vskip 2ex

\begin{quote}
	It’s worth considering whether TERFs, like certain strains of the altright, might be defined less by their political ideology (however noxious) and more by a complex, frankly fascinating relationship to trolling, on which it will be for future anthropologists, having solved the problem of digital ethnography, to elaborate. 54
\end{quote}
\vskip 2ex

Germaine Greer, \emph{The Female Eunuch} (1974) (second-wave, transphobic feminism).
\vskip 2ex

Robin Morgan, editor of the widely influential 1970 anthology \emph{Sisterhood Is Powerful}: drag like blackfacing
\begin{quote}
	“We know what’s at work when whites wear blackface; the same thing is at work when men wear drag.” (\emph{Lesbian Tide}, 1973; apropos transsexual folk singer Beth Elliott at West Coast Lesbian Conference of 1973) 55
\end{quote}
\vskip 2ex

second wave feminism, expanding the scope of feminist critique to everyday life, notably sexuality --- \emph{lesbianism}
\begin{quote}
	In expanding the scope of feminist critique to the terrain of everyday life---a move which produced a characteristically muscular brand of theory that rivaled any Marxist’s notes on capitalism---the second wave had inadvertently painted itself into a corner. If, as radical feminist theories claimed, patriarchy had infested not just legal, cultural, and economic spheres but the psychic lives of \emph{women themselves}, then feminist revolution could only be achieved by combing constantly through the fibrils of one’s consciousness for every last trace of male supremacy---a kind of political nitpicking, as it were. And nowhere was this more urgent, or more difficult, than the bedroom.\\[1ex] 
	Fighting tirelessly for the notion that sex was fair game for political critique, radical feminists were now faced with the prospect of putting their mouths where their money had been. Hence Atkinson’s famous slogan: \textbf{“Feminism is the theory, lesbianism is the practice.”} This was the political climate in which both Elliott and Morgan, as a transsexual woman and a suspected heterosexual woman, respectively, could find their statuses as legitimate subjects of feminist politics threatened by the incipient enshrining, among some radical feminists, of something called \textbf{lesbianism} as the preferred aesthetic form for mediating between individual subjects and the history they were supposed to be making--—call these \textbf{the personal and the political}. 57f.
\end{quote}
\vskip 2ex

lesbianism $\longrightarrow$ trans-exclusionary radical feminism
\begin{quote}
	there is a historical line to be traced from political lesbianism, as a specific, by no means dominant tendency within radical feminism, to the contemporary phenomenon we’ve taken to calling trans-exclusionary radical feminism
\end{quote}
\vskip 2ex

Sheila Jeffreys, \emph{Gender Hurts. A Feminist Analysis of the Politics of Transgenderism} (Routledge 2014): transgenderism as a ruthless appropriation of women’s experience and existence
\begin{quote}
	Like many TERFs, she believes \textbf{that trans women’s cheap imitations of femininity (as she imagines them) reproduce the same harmful stereotypes through which women are subordinated in the first place}. “Transgenderism on the part of men,” Jeffreys writes in her 2014 book \emph{Gender Hurts}, “can be seen as \textbf{a ruthless appropriation of women’s experience and existence}.”\\[1ex] 
	She is also fond of citing sexological literature that classifies transgenderism as a paraphilia. It is a favorite claim among TERFs like Jeffreys that transgender women are gropey interlopers, sick voyeurs conspiring to infiltrate women-only spaces and conduct the greatest panty raid in military history. 57
\end{quote}
\vskip 2ex

\begin{quote}
	Indeed, at least among lesbians, trans-exclusionary radical feminism might best be understood as gay panic, girl-on-girl edition. [\dots\hspace{-0.3ex}]\\[1ex]
	\textbf{trans-exclusionary feminism has inherited political lesbianism’s dread of desire’s ungovernability}. The traditional subject of gay panic, be he a US senator or just a member of the House, is a subject menaced by his own politically compromising desires: to preserve himself, he projects these desires onto another, whom he may now legislate or gay-bash out of existence. The political lesbian, too, is a subject stuck between the rock of politics and desire’s hard place. 57f.
\end{quote}
\vskip 2ex

lesbianism: not an innate identity, but an act of political will --- being a lesbian was about what got you woke, not wet
\begin{quote}
	As Jeffreys put it in 2015, speaking to the Lesbian History Group in London, political lesbianism was intended as a solution to the all-too-real cognitive dissonance produced by heterosexual feminism: “Why go to all these meetings where you’re creating all this wonderful theory and politics, and then you go home to, in my case, Dave, and you’re sitting there, you know, in front of the telly, and thinking, ‘It’s weird. This feels weird.’”\\[1ex] 
	But \textbf{true separatism doesn’t stop at leaving your husband}. It proceeds, with paranoid rigor, to purge the apartments of the mind of anything remotely connected to patriarchy. \textbf{Desire is no exception}.\\[1ex] 
	\textbf{Political lesbianism is founded on the belief that even desire becomes pliable at high enough temperatures}. For Jeffreys and her comrades, \textbf{lesbianism was not an innate identity, but an act of political will}. This was a world in which \textbf{biology was not destiny}, a world where \textbf{being a lesbian was about what got you woke, not wet}. 58
\end{quote}
\vskip 2ex

We are separatists from our own bodies
\begin{quote}
	It seems never to have occurred to Jeffreys that some of us “transgenders,” as she likes to call us, might opt to transition precisely in order to escape from the penitentiary she takes heterosexuality to be. \textbf{It is a supreme irony of feminist history that there is no woman more woman-identified than a gay trans girl like me}, and that Beth Elliott and her sisters were the OG political lesbians: \textbf{women who had walked away from both the men in their lives and the men whose lives they’d been living}. \textbf{We are separatists from our own bodies.} We are militants of so fine a caliber that we regularly take steps to poison the world’s supply of male biology. 58
\end{quote}
\vskip 2ex

\begin{quote}
	But let’s keep things in perspective. Because of Jeffreys, a few women in the Seventies got haircuts. Because of us, there are literally fewer men on the planet. 58
\end{quote}
\vskip 2ex


















\end{document}